% Options for packages loaded elsewhere
\PassOptionsToPackage{unicode}{hyperref}
\PassOptionsToPackage{hyphens}{url}
%
\documentclass[
]{article}
\usepackage{amsmath,amssymb}
\usepackage{iftex}
\ifPDFTeX
  \usepackage[T1]{fontenc}
  \usepackage[utf8]{inputenc}
  \usepackage{textcomp} % provide euro and other symbols
\else % if luatex or xetex
  \usepackage{unicode-math} % this also loads fontspec
  \defaultfontfeatures{Scale=MatchLowercase}
  \defaultfontfeatures[\rmfamily]{Ligatures=TeX,Scale=1}
\fi
\usepackage{lmodern}
\ifPDFTeX\else
  % xetex/luatex font selection
\fi
% Use upquote if available, for straight quotes in verbatim environments
\IfFileExists{upquote.sty}{\usepackage{upquote}}{}
\IfFileExists{microtype.sty}{% use microtype if available
  \usepackage[]{microtype}
  \UseMicrotypeSet[protrusion]{basicmath} % disable protrusion for tt fonts
}{}
\makeatletter
\@ifundefined{KOMAClassName}{% if non-KOMA class
  \IfFileExists{parskip.sty}{%
    \usepackage{parskip}
  }{% else
    \setlength{\parindent}{0pt}
    \setlength{\parskip}{6pt plus 2pt minus 1pt}}
}{% if KOMA class
  \KOMAoptions{parskip=half}}
\makeatother
\usepackage{xcolor}
\usepackage[margin=1in]{geometry}
\usepackage{color}
\usepackage{fancyvrb}
\newcommand{\VerbBar}{|}
\newcommand{\VERB}{\Verb[commandchars=\\\{\}]}
\DefineVerbatimEnvironment{Highlighting}{Verbatim}{commandchars=\\\{\}}
% Add ',fontsize=\small' for more characters per line
\usepackage{framed}
\definecolor{shadecolor}{RGB}{248,248,248}
\newenvironment{Shaded}{\begin{snugshade}}{\end{snugshade}}
\newcommand{\AlertTok}[1]{\textcolor[rgb]{0.94,0.16,0.16}{#1}}
\newcommand{\AnnotationTok}[1]{\textcolor[rgb]{0.56,0.35,0.01}{\textbf{\textit{#1}}}}
\newcommand{\AttributeTok}[1]{\textcolor[rgb]{0.13,0.29,0.53}{#1}}
\newcommand{\BaseNTok}[1]{\textcolor[rgb]{0.00,0.00,0.81}{#1}}
\newcommand{\BuiltInTok}[1]{#1}
\newcommand{\CharTok}[1]{\textcolor[rgb]{0.31,0.60,0.02}{#1}}
\newcommand{\CommentTok}[1]{\textcolor[rgb]{0.56,0.35,0.01}{\textit{#1}}}
\newcommand{\CommentVarTok}[1]{\textcolor[rgb]{0.56,0.35,0.01}{\textbf{\textit{#1}}}}
\newcommand{\ConstantTok}[1]{\textcolor[rgb]{0.56,0.35,0.01}{#1}}
\newcommand{\ControlFlowTok}[1]{\textcolor[rgb]{0.13,0.29,0.53}{\textbf{#1}}}
\newcommand{\DataTypeTok}[1]{\textcolor[rgb]{0.13,0.29,0.53}{#1}}
\newcommand{\DecValTok}[1]{\textcolor[rgb]{0.00,0.00,0.81}{#1}}
\newcommand{\DocumentationTok}[1]{\textcolor[rgb]{0.56,0.35,0.01}{\textbf{\textit{#1}}}}
\newcommand{\ErrorTok}[1]{\textcolor[rgb]{0.64,0.00,0.00}{\textbf{#1}}}
\newcommand{\ExtensionTok}[1]{#1}
\newcommand{\FloatTok}[1]{\textcolor[rgb]{0.00,0.00,0.81}{#1}}
\newcommand{\FunctionTok}[1]{\textcolor[rgb]{0.13,0.29,0.53}{\textbf{#1}}}
\newcommand{\ImportTok}[1]{#1}
\newcommand{\InformationTok}[1]{\textcolor[rgb]{0.56,0.35,0.01}{\textbf{\textit{#1}}}}
\newcommand{\KeywordTok}[1]{\textcolor[rgb]{0.13,0.29,0.53}{\textbf{#1}}}
\newcommand{\NormalTok}[1]{#1}
\newcommand{\OperatorTok}[1]{\textcolor[rgb]{0.81,0.36,0.00}{\textbf{#1}}}
\newcommand{\OtherTok}[1]{\textcolor[rgb]{0.56,0.35,0.01}{#1}}
\newcommand{\PreprocessorTok}[1]{\textcolor[rgb]{0.56,0.35,0.01}{\textit{#1}}}
\newcommand{\RegionMarkerTok}[1]{#1}
\newcommand{\SpecialCharTok}[1]{\textcolor[rgb]{0.81,0.36,0.00}{\textbf{#1}}}
\newcommand{\SpecialStringTok}[1]{\textcolor[rgb]{0.31,0.60,0.02}{#1}}
\newcommand{\StringTok}[1]{\textcolor[rgb]{0.31,0.60,0.02}{#1}}
\newcommand{\VariableTok}[1]{\textcolor[rgb]{0.00,0.00,0.00}{#1}}
\newcommand{\VerbatimStringTok}[1]{\textcolor[rgb]{0.31,0.60,0.02}{#1}}
\newcommand{\WarningTok}[1]{\textcolor[rgb]{0.56,0.35,0.01}{\textbf{\textit{#1}}}}
\usepackage{graphicx}
\makeatletter
\def\maxwidth{\ifdim\Gin@nat@width>\linewidth\linewidth\else\Gin@nat@width\fi}
\def\maxheight{\ifdim\Gin@nat@height>\textheight\textheight\else\Gin@nat@height\fi}
\makeatother
% Scale images if necessary, so that they will not overflow the page
% margins by default, and it is still possible to overwrite the defaults
% using explicit options in \includegraphics[width, height, ...]{}
\setkeys{Gin}{width=\maxwidth,height=\maxheight,keepaspectratio}
% Set default figure placement to htbp
\makeatletter
\def\fps@figure{htbp}
\makeatother
\setlength{\emergencystretch}{3em} % prevent overfull lines
\providecommand{\tightlist}{%
  \setlength{\itemsep}{0pt}\setlength{\parskip}{0pt}}
\setcounter{secnumdepth}{-\maxdimen} % remove section numbering
\ifLuaTeX
  \usepackage{selnolig}  % disable illegal ligatures
\fi
\usepackage{bookmark}
\IfFileExists{xurl.sty}{\usepackage{xurl}}{} % add URL line breaks if available
\urlstyle{same}
\hypersetup{
  pdftitle={Project1},
  pdfauthor={Kate Brown},
  hidelinks,
  pdfcreator={LaTeX via pandoc}}

\title{Project1}
\author{Kate Brown}
\date{2024-09-03}

\begin{document}
\maketitle

\begin{Shaded}
\begin{Highlighting}[]
\FunctionTok{library}\NormalTok{(}\StringTok{"here"}\NormalTok{)}
\end{Highlighting}
\end{Shaded}

\begin{verbatim}
## here() starts at C:/Users/19143/OneDrive - Johns Hopkins/Documents/BSTA776projects
\end{verbatim}

\begin{Shaded}
\begin{Highlighting}[]
\FunctionTok{library}\NormalTok{(}\StringTok{"tidyverse"}\NormalTok{)}
\end{Highlighting}
\end{Shaded}

\begin{verbatim}
## -- Attaching core tidyverse packages ------------------------ tidyverse 2.0.0 --
## v dplyr     1.1.4     v readr     2.1.5
## v forcats   1.0.0     v stringr   1.5.1
## v ggplot2   3.5.1     v tibble    3.2.1
## v lubridate 1.9.3     v tidyr     1.3.1
## v purrr     1.0.2
\end{verbatim}

\begin{verbatim}
## -- Conflicts ------------------------------------------ tidyverse_conflicts() --
## x dplyr::filter() masks stats::filter()
## x dplyr::lag()    masks stats::lag()
## i Use the conflicted package (<http://conflicted.r-lib.org/>) to force all conflicts to become errors
\end{verbatim}

\begin{Shaded}
\begin{Highlighting}[]
\CommentTok{\# tests if a directory named "data" exists locally}
\ControlFlowTok{if}\NormalTok{ (}\SpecialCharTok{!}\FunctionTok{dir.exists}\NormalTok{(}\FunctionTok{here}\NormalTok{(}\StringTok{"data"}\NormalTok{))) \{}
    \FunctionTok{dir.create}\NormalTok{(}\FunctionTok{here}\NormalTok{(}\StringTok{"data"}\NormalTok{))}
\NormalTok{\}}

\CommentTok{\# saves data only once (not each time you knit a R Markdown)}
\ControlFlowTok{if}\NormalTok{ (}\SpecialCharTok{!}\FunctionTok{file.exists}\NormalTok{(}\FunctionTok{here}\NormalTok{(}\StringTok{"data"}\NormalTok{, }\StringTok{"chocolate.RDS"}\NormalTok{))) \{}
\NormalTok{    url\_csv }\OtherTok{\textless{}{-}} \StringTok{"https://raw.githubusercontent.com/rfordatascience/tidytuesday/master/data/2022/2022{-}01{-}18/chocolate.csv"}
\NormalTok{    chocolate }\OtherTok{\textless{}{-}}\NormalTok{ readr}\SpecialCharTok{::}\FunctionTok{read\_csv}\NormalTok{(url\_csv)}

    \CommentTok{\# save the file to RDS objects}
    \FunctionTok{saveRDS}\NormalTok{(chocolate, }\AttributeTok{file =} \FunctionTok{here}\NormalTok{(}\StringTok{"data"}\NormalTok{, }\StringTok{"chocolate.RDS"}\NormalTok{))}
\NormalTok{\}}
\end{Highlighting}
\end{Shaded}

\begin{Shaded}
\begin{Highlighting}[]
\NormalTok{chocolate }\OtherTok{=} \FunctionTok{readRDS}\NormalTok{(}\FunctionTok{here}\NormalTok{(}\StringTok{"data"}\NormalTok{, }\StringTok{"chocolate.RDS"}\NormalTok{))}
\FunctionTok{as\_tibble}\NormalTok{(chocolate)}
\end{Highlighting}
\end{Shaded}

\begin{verbatim}
## # A tibble: 2,530 x 10
##      ref company_manufacturer company_location review_date
##    <dbl> <chr>                <chr>                  <dbl>
##  1  2454 5150                 U.S.A.                  2019
##  2  2458 5150                 U.S.A.                  2019
##  3  2454 5150                 U.S.A.                  2019
##  4  2542 5150                 U.S.A.                  2021
##  5  2546 5150                 U.S.A.                  2021
##  6  2546 5150                 U.S.A.                  2021
##  7  2542 5150                 U.S.A.                  2021
##  8   797 A. Morin             France                  2012
##  9   797 A. Morin             France                  2012
## 10  1011 A. Morin             France                  2013
## # i 2,520 more rows
## # i 6 more variables: country_of_bean_origin <chr>,
## #   specific_bean_origin_or_bar_name <chr>, cocoa_percent <chr>,
## #   ingredients <chr>, most_memorable_characteristics <chr>, rating <dbl>
\end{verbatim}

\begin{Shaded}
\begin{Highlighting}[]
\FunctionTok{glimpse}\NormalTok{(chocolate)}
\end{Highlighting}
\end{Shaded}

\begin{verbatim}
## Rows: 2,530
## Columns: 10
## $ ref                              <dbl> 2454, 2458, 2454, 2542, 2546, 2546, 2~
## $ company_manufacturer             <chr> "5150", "5150", "5150", "5150", "5150~
## $ company_location                 <chr> "U.S.A.", "U.S.A.", "U.S.A.", "U.S.A.~
## $ review_date                      <dbl> 2019, 2019, 2019, 2021, 2021, 2021, 2~
## $ country_of_bean_origin           <chr> "Tanzania", "Dominican Republic", "Ma~
## $ specific_bean_origin_or_bar_name <chr> "Kokoa Kamili, batch 1", "Zorzal, bat~
## $ cocoa_percent                    <chr> "76%", "76%", "76%", "68%", "72%", "8~
## $ ingredients                      <chr> "3- B,S,C", "3- B,S,C", "3- B,S,C", "~
## $ most_memorable_characteristics   <chr> "rich cocoa, fatty, bready", "cocoa, ~
## $ rating                           <dbl> 3.25, 3.50, 3.75, 3.00, 3.00, 3.25, 3~
\end{verbatim}

\begin{Shaded}
\begin{Highlighting}[]
\FunctionTok{library}\NormalTok{(ggplot2)}
\FunctionTok{library}\NormalTok{(dplyr)}
\end{Highlighting}
\end{Shaded}

\begin{enumerate}
\def\labelenumi{\arabic{enumi}.}
\tightlist
\item
  Make a histogram of the rating scores to visualize the overall
  distribution of scores. Change the number of bins from the default to
  10, 15, 20, and 25. Pick on the one that you think looks the best.
  Explain what the difference is when you change the number of bins and
  explain why you picked the one you did.
\end{enumerate}

\begin{Shaded}
\begin{Highlighting}[]
\FunctionTok{ggplot}\NormalTok{(chocolate, }\FunctionTok{aes}\NormalTok{(}\AttributeTok{x=}\NormalTok{rating)) }\SpecialCharTok{+} \FunctionTok{geom\_histogram}\NormalTok{(}
\NormalTok{) }\CommentTok{\#default is 30}
\end{Highlighting}
\end{Shaded}

\begin{verbatim}
## `stat_bin()` using `bins = 30`. Pick better value with `binwidth`.
\end{verbatim}

\includegraphics{project1_files/figure-latex/unnamed-chunk-5-1.pdf}

\begin{Shaded}
\begin{Highlighting}[]
\FunctionTok{ggplot}\NormalTok{(chocolate, }\FunctionTok{aes}\NormalTok{(}\AttributeTok{x=}\NormalTok{rating)) }\SpecialCharTok{+} \FunctionTok{geom\_histogram}\NormalTok{(}\AttributeTok{bins =} \DecValTok{10}\NormalTok{)}
\end{Highlighting}
\end{Shaded}

\includegraphics{project1_files/figure-latex/unnamed-chunk-5-2.pdf}

\begin{Shaded}
\begin{Highlighting}[]
\FunctionTok{ggplot}\NormalTok{(chocolate, }\FunctionTok{aes}\NormalTok{(}\AttributeTok{x=}\NormalTok{rating)) }\SpecialCharTok{+} \FunctionTok{geom\_histogram}\NormalTok{(}
\AttributeTok{bins =} \DecValTok{15}\NormalTok{)}
\end{Highlighting}
\end{Shaded}

\includegraphics{project1_files/figure-latex/unnamed-chunk-5-3.pdf}

\begin{Shaded}
\begin{Highlighting}[]
\FunctionTok{ggplot}\NormalTok{(chocolate, }\FunctionTok{aes}\NormalTok{(}\AttributeTok{x=}\NormalTok{rating)) }\SpecialCharTok{+} \FunctionTok{geom\_histogram}\NormalTok{(}
\AttributeTok{bins=}\DecValTok{20}\NormalTok{)}
\end{Highlighting}
\end{Shaded}

\includegraphics{project1_files/figure-latex/unnamed-chunk-5-4.pdf}

\begin{Shaded}
\begin{Highlighting}[]
\FunctionTok{ggplot}\NormalTok{(chocolate, }\FunctionTok{aes}\NormalTok{(}\AttributeTok{x=}\NormalTok{rating)) }\SpecialCharTok{+} \FunctionTok{geom\_histogram}\NormalTok{(}\AttributeTok{bins=}\DecValTok{25}
\NormalTok{)}
\end{Highlighting}
\end{Shaded}

\includegraphics{project1_files/figure-latex/unnamed-chunk-5-5.pdf}
Changing the number of bins changes how many bars you will see on the
histogram. I choose 15 bins because it shows that there are 2 rating
numbers that were higher than the rest of the other ratings given.

\begin{enumerate}
\def\labelenumi{\arabic{enumi}.}
\setcounter{enumi}{1}
\tightlist
\item
  Consider the countries where the beans originated from. How many
  reviews come from each country of bean origin?
\end{enumerate}

\begin{Shaded}
\begin{Highlighting}[]
\FunctionTok{table}\NormalTok{(chocolate}\SpecialCharTok{$}\NormalTok{country\_of\_bean\_origin)}
\end{Highlighting}
\end{Shaded}

\begin{verbatim}
## 
##             Australia                Belize                 Blend 
##                     3                    76                   156 
##               Bolivia                Brazil                 Burma 
##                    80                    78                     1 
##              Cameroon                 China              Colombia 
##                     3                     1                    79 
##                 Congo            Costa Rica                  Cuba 
##                    11                    43                    12 
##    Dominican Republic              DR Congo               Ecuador 
##                   226                     1                   219 
##           El Salvador                  Fiji                 Gabon 
##                     6                    16                     1 
##                 Ghana               Grenada             Guatemala 
##                    41                    19                    62 
##                 Haiti              Honduras                 India 
##                    30                    25                    35 
##             Indonesia           Ivory Coast               Jamaica 
##                    20                     7                    24 
##               Liberia            Madagascar              Malaysia 
##                     3                   177                     8 
##            Martinique                Mexico             Nicaragua 
##                     1                    55                   100 
##               Nigeria                Panama      Papua New Guinea 
##                     3                     9                    50 
##                  Peru           Philippines              Principe 
##                   244                    24                     1 
##           Puerto Rico                 Samoa              Sao Tome 
##                     7                     3                    14 
##   Sao Tome & Principe          Sierra Leone       Solomon Islands 
##                     2                     4                    10 
##             Sri Lanka             St. Lucia St.Vincent-Grenadines 
##                     2                    10                     1 
##              Sulawesi               Sumatra              Suriname 
##                     1                     1                     1 
##                Taiwan              Tanzania              Thailand 
##                     2                    79                     5 
##                Tobago                  Togo              Trinidad 
##                     2                     3                    42 
##                U.S.A.                Uganda               Vanuatu 
##                    33                    19                    13 
##             Venezuela               Vietnam 
##                   253                    73
\end{verbatim}

\begin{Shaded}
\begin{Highlighting}[]
\FunctionTok{sum}\NormalTok{(chocolate}\SpecialCharTok{$}\NormalTok{country\_of\_bean\_origin }\SpecialCharTok{==} \StringTok{"Vietnam"}\NormalTok{)}
\end{Highlighting}
\end{Shaded}

\begin{verbatim}
## [1] 73
\end{verbatim}

\begin{enumerate}
\def\labelenumi{\arabic{enumi}.}
\setcounter{enumi}{2}
\tightlist
\item
  What is average rating scores from reviews of chocolate bars that have
  Ecuador as country\_of\_bean\_origin in this dataset? For this same
  set of reviews, also calculate (1) the total number of reviews and (2)
  the standard deviation of the rating scores. Your answer should be a
  new data frame with these three summary statistics in three columns.
  Label the name of these columns mean, sd, and total.
\end{enumerate}

\begin{Shaded}
\begin{Highlighting}[]
\NormalTok{subset }\OtherTok{=}\NormalTok{ chocolate}\SpecialCharTok{$}\NormalTok{country\_of\_bean\_origin }\SpecialCharTok{==} \StringTok{"Ecuador"}
\FunctionTok{sum}\NormalTok{(subset)}
\end{Highlighting}
\end{Shaded}

\begin{verbatim}
## [1] 219
\end{verbatim}

\end{document}
